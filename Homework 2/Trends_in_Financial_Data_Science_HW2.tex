\documentclass[11pt]{article}

\usepackage{amsmath,amsthm,amssymb}

%%%%% Matrix stretcher
% use it as:
%\begin{pmatrix}[1.5]
% ...
\makeatletter
\renewcommand*\env@matrix[1][\arraystretch]{%
  \edef\arraystretch{#1}%
  \hskip -\arraycolsep
  \let\@ifnextchar\new@ifnextchar
  \array{*\c@MaxMatrixCols c}}
\makeatother
%%%%%%%%%%%%%%%%%%%%%%%%%%

\newcommand\extrafootertext[1]{%
    \bgroup
    \renewcommand\thefootnote{\fnsymbol{footnote}}%
    \renewcommand\thempfootnote{\fnsymbol{mpfootnote}}%
    \footnotetext[0]{#1}%
    \egroup
}


%%%%%%%%%%%%% Colors %%%%%%%%%%%%%
\usepackage[dvipsnames]{xcolor}

\definecolor{C0}{HTML}{1d1d1d}
\definecolor{C1}{HTML}{1e3668}
\definecolor{C2}{HTML}{199d8b}
\definecolor{C3}{HTML}{d52f4c}
\definecolor{C4}{HTML}{5ab2d6}
\definecolor{C5}{HTML}{ffb268}
\definecolor{C6}{HTML}{ff7300} % for commenting - {fire orange}dd571c
\definecolor{C7}{HTML}{777b7e} % for remarks - {steel grey}
\color{C0}
%%%%%%%%%%%%%%%%%%%%%%%%%%%%%%%%



%%%%%%%%%%%%% Fonts %%%%%%%%%%%%% 
%\usepackage{fontspec}
\usepackage[no-math]{fontspec} % for text

\emergencystretch=8pt
\hyphenpenalty=1000 % default 50
\tolerance=800      % default 200
%\righthyphenmin=4
%\lefthyphenmin=4

%%% Text Font: Vollkorn + Math Font: Latin Modern (default) %%%
\setmainfont{Vollkorn}[
UprightFont = Vollkorn-Regular,
ItalicFont =Vollkorn-Italic, 
BoldItalicFont={Vollkorn-BoldItalic},
BoldFont = Vollkorn-Bold,
RawFeature=+lnum,
WordSpace=1.7,
] 

%%% We need this for math font packages other than latin modern %%%
% \usepackage{unicode-math}        % for math

%%% Text Font: Palatino + Math Font: Asana-Math %%%
%\setmainfont{Palatino}[
%BoldFont = Palatino-Bold,
%ItalicFont = Palatino-Italic,
%BoldItalicFont={Palatino-BoldItalic},
%RawFeature=+lnum,
%WordSpace=1.7,
%]
%\setmathfont{asana-math}

%%% Text Font: Arno Pro + Math Font: Minion Pro %%%
%\setmainfont{Arno Pro}[
%UprightFont = *-Regular,
%ItalicFont = Vollkorn-Italic, 
%BoldItalicFont={*-BoldItalic},
%BoldFont = *-Bold,
%RawFeature=+lnum,
%WordSpace=1.7,
%Scale= 1.1
%] 
% Minion Pro is too expensive

%%% Math Fonts %%%
%\setmathfont{Vollkorn}
%\setmathfont{Latin Modern Math}
%\setmathfont{TeX Gyre Pagella Math}
%\setmathfont{TeX Gyre Termes Math}
%\setmathfont{TeX Gyre DejaVu Math}
%\setmathfont[Scale=MatchLowercase]{DejaVu Math TeX Gyre}
%\setmathfont{XITS Math}
%\setmathfont{Libertinus Math}
%\setmathfont[Scale=MatchUppercase]{Asana Math}
%\setmathfont{STIX Two Math}

%\usepackage{kpfonts-otf}
%\setmathfont{KpMath-Regular.otf}[version=regular]
%\setmathfont{KpMath-Bold.otf}[version=bold]
%\setmathfont{KpMath-Semibold.otf}[version=semibold]
%\setmathfont{KpMath-Sans.otf}[version=sans]
%\setmathfont{KpMath-Light.otf}[version=light]


%%% CJK Fonts %%%
\usepackage[scale=.78]{luatexja-fontspec}
\setmainjfont{BabelStone Han}[AutoFakeBold]
%%%%%%%%%%%%%%%%%%%%%%%%%%%%%%%


% This package simplifies the insertion of external multi-page PDF or PS documents.
\usepackage{pdfpages}

% cref
\usepackage{hyperref}
\hypersetup{
    colorlinks=true,
    linkcolor=C4,
    filecolor=magenta,      
    urlcolor=cyan,
    }

\usepackage[nameinlink,noabbrev,capitalize]{cleveref}
% \crefname{ineq}{}{}
% \crefname{equation}{}{}
% \creflabelformat{ineq}{#2{\textup{(1)}}#3}
% \creflabelformat{equation}{#2\textup{(#1)}#3}

%%%%%%%%%%%%% Environments %%%%%%%%%%%%%%%%
%amsthm has three separate predefined styles:	
%
%\theoremstyle{plain} is the default. it sets the text in italic and adds extra space above and below the \newtheorems listed below it in the input. it is recommended for theorems, corollaries, lemmas, propositions, conjectures, criteria, and (possibly; depends on the subject area) algorithms.
%
%\theoremstyle{definition} adds extra space above and below, but sets the text in roman. it is recommended for definitions, conditions, problems, and examples; i've alse seen it used for exercises.
%
%\theoremstyle{remark} is set in roman, with no additional space above or below. it is recommended for remarks, notes, notation, claims, summaries, acknowledgments, cases, and conclusions.

%%%  theorem-like environment %%%
\theoremstyle{plain} % default theorem style
\newtheorem{theorem}{Theorem}[section]
\newtheorem{assumption}[theorem]{Assumption}
\newtheorem{lemma}[theorem]{Lemma}
\newtheorem{corollary}[theorem]{Corollary}
\newtheorem{proposition}[theorem]{Proposition}
\newtheorem{property}[theorem]{Property}

\newtheorem{definition}[theorem]{Definition}

%%% definition-like environment %%%
%\theoremstyle{definition}
\newtheorem{example}[theorem]{Example}
\newtheorem{problem}[theorem]{Problem}


%%% framed package is great %%%
\usepackage{framed}
\newenvironment{solution}
{\color{C2}\normalfont\begin{framed}\begingroup\textbf{Solution:} }
  {\endgroup\end{framed}}

\newtheoremstyle{remark}% name of the style to be used
  {}% measure of space to leave above the theorem. E.g.: 3pt
  {}% measure of space to leave below the theorem. E.g.: 3pt
  {\color{C3}}% name of font to use in the body of the theorem
  {}% measure of space to indent
  {\color{C3}\bfseries}% name of head font
  {.}% punctuation between head and body
  { }% space after theorem head; " " = normal interword space
  {}
\theoremstyle{remark}
\newtheorem{remarkx}[theorem]{Remark}
\newenvironment{remark}
  {\pushQED{\qed}\renewcommand{\qedsymbol}{$\triangle$}\remarkx}
  {\popQED\endremarkx}
  
\newenvironment{point}
  {\O~~}
  {}

\usepackage{thmtools}
\usepackage{thm-restate}
%%%%%%%%%%%%%%%%%%%%%%%%%%%%%%%%%%%%


% This package is for the long equal sign \xlongequal{}
\usepackage{extarrows}


% Page Formatting
\usepackage[
    paper=a3paper,
    inner=22mm,         % Inner margin
    outer=22mm,         % Outer margin
    bindingoffset=0mm, % Binding offset
    top=28mm,           % Top margin
    bottom=22mm,        % Bottom margin
    %showframe,         % show how the type block is set on the page
]{geometry}

\setlength{\parindent}{0em}
\setlength{\parskip}{.7em}


\usepackage{tikz}
\usepackage{graphicx}
\usepackage{enumitem}
\setlist{topsep=0pt}

\usepackage{bm}

\usepackage[font=scriptsize,labelfont=bf]{caption}
\usepackage{listings}
\lstset{basicstyle=\ttfamily,breaklines=true}
% \setlength{\parskip}{1em}
% \setlength{\parindent}{0em}
\usepackage{dsfont}
\newcommand{\bOne}{\mathds{1}}
\newcommand{\PP}{\mathbb{P}}
\newcommand{\EE}{\mathbb{E}}
\newcommand{\VV}{\mathbb{V}}
\newcommand{\CoV}{\operatorname{Co\mathbb{V}}}

% header
\usepackage{fancyhdr}
\pagestyle{fancy}
\fancyhead{}
\fancyhead[L]{\small   \bfseries Homework}
\fancyhead[C]{\small   \bfseries Fall 2023}
\fancyhead[R]{\small   \bfseries Zhou}


\begin{document}

\begin{center}
  \text{\Large{Black-Litterman-Bayes, Kalman Filter, ICA
    }}

  {\text{Kaiwen Zhou}}
\end{center}
\vspace{2em}

\tableofcontents

\section*{Useful Lemma}
\begin{lemma}\label{lemma: gaussian}
  If a multivariate normal random variable $\boldsymbol{\theta}$ has density $p(\boldsymbol{\theta})$ and
$$
-2 \log p(\boldsymbol{\theta})=\boldsymbol{\theta}^\top \boldsymbol{H} \boldsymbol{\theta}-2 \boldsymbol{\eta}^\top \boldsymbol{\theta}+(\text { terms without } \boldsymbol{\theta})
$$
then $\mathbb{V}[\boldsymbol{\theta}]=\boldsymbol{H}^{-1}$ and $\mathbb{E} [\boldsymbol{\theta}] =\boldsymbol{H}^{-1} \boldsymbol{\eta}$.
\end{lemma}
\begin{solution}
  For $\boldsymbol{H}$ symmetric, we have
  $$
  \boldsymbol{\theta}^\top \boldsymbol{H} \boldsymbol{\theta}-2 \boldsymbol{v}^\top \boldsymbol{H} \boldsymbol{\theta}=(\boldsymbol{\theta}-\boldsymbol{v})^\top \boldsymbol{H}(\boldsymbol{\theta}-\boldsymbol{v})-\boldsymbol{v}^\top \boldsymbol{H} \boldsymbol{v}
  $$
  Set $\boldsymbol{v}= \boldsymbol{H}^{-1} \boldsymbol{\eta}$ in the above equation, we obtain
  $$
  -2 \log p(\boldsymbol{\theta})=\boldsymbol{\theta}^\top \boldsymbol{H} \boldsymbol{\theta}-2 \boldsymbol{\eta}^\top \boldsymbol{\theta}+(\text { terms without } \boldsymbol{\theta}) = (\boldsymbol{\theta}-\boldsymbol{H}^{-1} \boldsymbol{\eta})^\top \boldsymbol{H}(\boldsymbol{\theta}-\boldsymbol{H}^{-1} \boldsymbol{\eta}) + (\text { terms without } \boldsymbol{\theta})
  $$
  Therefore, we must have $\mathbb{V}[\boldsymbol{\theta}]=\boldsymbol{H}^{-1}$ and $\mathbb{E} [\boldsymbol{\theta}] =\boldsymbol{H}^{-1} \boldsymbol{\eta}$.
\end{solution}

\section{Topic: Black-Litterman-Bayes}
\begin{problem}
This question refers to the article "On The Bayesian Interpretation Of
Black-Litterman" (Kolm and Ritter, EJOR 2017).
\begin{enumerate}[label=(\alph*)]
  \item Derive formulas (10)-(11) using the properties of the multivariate
        normal distribution in the slides "Bayesian Modeling: Introduction".
  \item \textbf{(Extra credit)} Derive formulas (22)-(26) using the same
        properties.
\end{enumerate}
\end{problem}
\begin{solution}
  Suppose we have $\boldsymbol{r} \sim N(\boldsymbol{\theta}, \boldsymbol{\Sigma})$, 
  and since Black and Litterman were motivated by
  the guiding principle that, in the absence of any sort of information/views
  which could constitute alpha over the benchmark, the optimization procedure
  should simply return the global CAPM equilibrium portfolio, with holdings
  denoted $\boldsymbol{h}_{e q}$. Hence in the absence of any views, and with
  prior mean equal to $\Pi$, the investor's model of the world is that
\begin{align}
  \boldsymbol{r} \sim \mathcal{N}(\boldsymbol{\theta}, \boldsymbol{\Sigma}) \  \text { and } \  \boldsymbol{\theta} \sim \mathcal{N}(\boldsymbol{\Pi}, \boldsymbol{C}) \quad \text{ where } \boldsymbol{r}, \boldsymbol{\theta},\boldsymbol{\Pi} \in \mathbb{R}^n, \boldsymbol{\Sigma}, \boldsymbol{C}\in \mathbb{R}^{n\times n}\label{eq:BLB returns}
\end{align}
A key aspect of the model is that the practitioner must also specify a level of
uncertainty or ``error bar'' for each view, which is assumed to be an independent
source of noise from the volatility already accounted for in a model such as
$\boldsymbol{\Sigma}$. This is expressed as the following:
\begin{align}
  \boldsymbol{P} \boldsymbol{\theta}=\boldsymbol{q}+\boldsymbol{\epsilon}^{(v)}, \quad \boldsymbol{\epsilon}^{(v)} \sim \mathcal{N}(0, \boldsymbol{\Omega}), \quad \boldsymbol{\Omega}=\operatorname{diag}\left(\omega_1, \ldots, \omega_k\right) \label{eq:BLB views}
\end{align}
where $\boldsymbol{P} \in \mathbb{R}^{k\times n}$, $\boldsymbol{\Omega} \in \mathbb{R}^{k\times k}$ and $\boldsymbol{q},\boldsymbol{\epsilon}^{(v)} \in \mathbb{R}^k$

  \begin{enumerate}[label=(\alph*)]
    \item In this question, we derive the mean $\boldsymbol{\nu}$ and covariance
    matrix $\boldsymbol{H}$ for the posterior.
    
    Since the posterior is proportional to the product of the likelihood and the
    prior, to simplify our computation, we neglect the constant coefficients of related probability density functions in
    our derivation.

    From \cref{eq:BLB views} and \cref{eq:BLB returns}, we have the likelihood function and the prior to be 
    $$
f(\boldsymbol{\theta}\mid \boldsymbol{q}) \propto \exp \left[-\frac{1}{2}(\boldsymbol{P} \boldsymbol{\theta}-\boldsymbol{q})^\top \boldsymbol{\Omega}^{-1}(\boldsymbol{P} \boldsymbol{\theta}-\boldsymbol{q})\right], f(\boldsymbol{\theta}) \propto \exp \left[-\frac{1}{2}(\boldsymbol{\theta} -\boldsymbol{\Pi})^\top \boldsymbol{\Sigma}^{-1}(\boldsymbol{\theta} -\boldsymbol{\Pi})\right]
$$
Leveraging the Bayes's formula, we have $f(\boldsymbol{\theta}\mid \boldsymbol{q}) \propto f(\boldsymbol{q} \mid \boldsymbol{\theta})f(\boldsymbol{\theta})$. It follows that
$$
-2\log f(\boldsymbol{\theta}\mid \boldsymbol{q}) \propto (\boldsymbol{P} \boldsymbol{\theta} -\boldsymbol{q})^\top \boldsymbol{\Omega}^{-1}(\boldsymbol{P} \boldsymbol{\theta}-\boldsymbol{q})+(\boldsymbol{\theta}-\boldsymbol{\Pi})^\top \mathbf{C}^{-1}(\boldsymbol{\theta}-\boldsymbol{\Pi}) \xlongequal[\text{ drop terms without $\boldsymbol{\theta}$}]{\text{ completing the squares }}\boldsymbol{\theta}^\top\left[\boldsymbol{P}^\top \boldsymbol{\Omega}^{-1} \mathbf{P}+\boldsymbol{C}^{-1}\right] \boldsymbol{\theta}-2\left(\boldsymbol{q}^\top \boldsymbol{\Omega}^{-1} \mathbf{P}+\boldsymbol{\Pi}^\top \mathbf{C}^{-1}\right) \boldsymbol{\theta}
$$
By \cref{lemma: gaussian}, we obtain
$$
\boldsymbol{\theta}\mid \boldsymbol{q} \sim \mathcal{N}\left(\boldsymbol{\nu}, \boldsymbol{H}^{-1}\right), \quad \boldsymbol{\nu}=\left[\boldsymbol{P}^\top \boldsymbol{\Omega}^{-1} \boldsymbol{P}+\boldsymbol{C}^{-1}\right]^{-1}\left[\boldsymbol{P}^\top \boldsymbol{\Omega}^{-1} \boldsymbol{q}+\boldsymbol{C}^{-1} \boldsymbol{\Pi}\right] \text{ and } \boldsymbol{H}^{-1}=\left[\boldsymbol{P}^\top \boldsymbol{\Omega}^{-1} \boldsymbol{P}+\boldsymbol{C}^{-1}\right]^{-1}
$$
    \item \textbf{(Extra credit)} Derive formulas (22)-(26) using the same
          properties.
  \end{enumerate}
\end{solution}



\section{Topic: Important Properties Of the Kalman Filter}
\begin{problem}
Choose three out of the four subproblems. It is optional to solve the other one
for extra credit.
\begin{enumerate}[label=(\alph*)]
  \item In class we derived the Kalman filter under the assumption that
        $\mathbb{E}\left[\mathbf{w}_t \mathbf{v}_t^\top\right]=\mathbf{0}$. Now,
        derive the Kalman filter under the assumption that
        $\mathbb{E}\left[\mathbf{w}_t \mathbf{v}_t^\top\right]=\mathbf{M}_t$.
  \item Using the notation from class, let us denote the error between the true
        and estimated states by the random variable
        $\tilde{\mathbf{x}}_t=\mathbf{x}_t-\hat{\mathbf{x}}_t$. We denote by
        $\mathbf{W}_t$ a known positive definite matrix. Show that the Kalman filter
        is the solution to the problem
        \begin{align}
          \min _{\hat{\mathbf{x}}_t} \mathbb{E}\left[\tilde{\mathbf{x}}_t^\top \mathbf{W}_t \tilde{\mathbf{x}}_t\right]\label{eq: kalman}
        \end{align}
  \item Now let us assume that $\mathbf{w}_t$ and $\mathbf{v}_t$ have zero mean,
        are uncorrelated with covariance matrices $\mathbf{Q}_t$ and $\mathbf{R}_t$,
        respectively (but they are no longer Gaussian). Show that the Kalman filter is
        the best linear solution to \cref{eq: kalman}. In other words, the Kalman
        filter is the best filter that is a linear combination of the measurements,
        $\mathbf{y}_t$.
  \item In class we derived the Kalman filter under the assumption that
        $\mathbf{w}_t$ and $\mathbf{v}_t$ are uncolored (i.e. each is serially
        uncorrelated).
        \begin{enumerate}[label=(\roman*)]
          \item Derive the Kalman filter under the assumption that
                $\mathbf{w}_t$ is a VAR(1) process with known system matrix.
          \item Derive the Kalman filter under the assumption that
                $\mathbf{v}_t$ is a $\operatorname{VAR}(1)$ process with known
                system matrix.
        \end{enumerate}
        Hint: For (d), both (i) and (ii) can be solved by properly augmenting
        the state equations. Can you find a solution to (ii) where one does not
        have to augment the state? Is it possible to do so for (i) - why, or why
        not?
\end{enumerate}

\end{problem}


\section{Topic: Kalman filter's Application on Financial Problems --- Pairs Trading}
\begin{problem}
In this question we consider a basic pairs trading strategy between two stocks
with prices $p_t^A$ and $p_t^B$ at time $t$. We denote the spread between them
by $s_t:=\log \left(p_t^A\right)-\log \left(p_t^B\right)$ and assume the spread
follows an Ornstein-Uhlenbeck process
$$
  d s_t=\kappa\left(\theta-s_t\right) d t+\sigma d B_t
$$
where $d B_t$ is a standard Brownian motion. In other words, the spread reverts
to its mean $\theta \in \mathbb{R}$ at the speed $\kappa \in \mathbb{R}_{+}$and
volatility $\sigma \in \mathbb{R}_{+}$.
\begin{enumerate}[label=(\alph*)]
  \item Show that the discrete time solution of (2) is Markovian, that is
        $$
          s_k=\mathbb{E}\left[s_k \mid s_{k-1}\right]+\varepsilon_k
        $$
        where $k=1,2, \ldots$, and $\varepsilon_k$ is a random process with zero
        mean and variance equal to $\sigma_{\varepsilon,
            k}^2=\mathbb{V}\left[s_k \mid s_{k-1}\right]$. (Hint: You can derive the
        discrete solution explicitly.)
  \item Propose a methodology for updating the parameters $\theta, \kappa$ using
        the Kalman filter and describe how you would use it to trade the stock pair.
  \item Test your methodology from (b) on simulated data. In particular, (i)
        simulate (2) from known parameters $\theta, \kappa$ and $\sigma$, and then
        (ii) use the Kalman filter to recover them. You do not need to implement the
        Kalman filter from scratch; you are welcome to use a Kalman implementation
        from a Python package such as pykalman. How do you obtain a good estimate of
        $\sigma$ ?
  \item Repeat the same experiment from (c), but this time simulate (2) first
        with $\kappa$ having the same value as above and then suddenly changing it to
        another value such that the half-life of the spread is $50 \%$ of its original
        value. How long does it take the Kalman filter to adjust? Can you make
        adjustment to your filter in order to speed up the time it takes the Kalman
        filter to adjust?
\end{enumerate}
Hint: For (c) and (d), think about how you are going to demonstrate the results
using appropriate graphs, etc.
\end{problem}

\section{Topic: Kalman filter's Application on Financial Problems --- Index Tracking Portfolios}

\begin{problem}
It is common in portfolio management to build so-called (index) tracking
portfolios. Let us assume we are observing the return of the S\&P 500 benchmark
index, $r_{b, t}$. Now, let us pick a subset of 50 stocks from the constituents
of this index. We will use these stocks to build a tracking portfolio for the
index. For example, this could be the 50 companies in the index with the largest
market cap. We denote the returns of these 50 stock by $\mathbf{r}_t \in
  \mathbb{R}^{50}$. The goal of finding a tracking portfolio is to find a dynamic
trading strategy of the 50 stocks such that $\boldsymbol{\beta}_t^\top
  \mathbf{r}_t \approx r_{b, t}$, where $\boldsymbol{\beta}_t$ denotes the
holdings of the tracking portfolio.
\begin{enumerate}[label=(\alph*)]
  \item In this part, we assume that the covariance matrix of returns of the
        stocks in the $\mathrm{S} \& \mathrm{P} 500, \boldsymbol{\Sigma}$, is given
        and constant through time. Find the portfolio of these 50 stocks that
        minimizes the tracking error to $r_{b, t}$, i.e. find the solution to
        $$
          \boldsymbol{\beta}_t^*=\operatorname{argmin}_{\boldsymbol{\beta}_t} \sqrt{\mathbb{V}\left[r_{b, t}-\boldsymbol{\beta}_t^\top \mathbf{r}_t\right]} .
        $$

        What specific property does $\boldsymbol{\beta}_t$ have here?
  \item In this part, we no longer assume that covariances amongst stocks are
        time invariant. Propose a solution to minimizing the tracking error using the
        Kalman filter.
  \item Download daily market data and create an example that illustrates your
        methodology. How does the tracking portfolio of the Kalman filter perform?
\end{enumerate}
Hint: For (c), compare your Kalman filter to a solution based on (a) where the
covariance matrix is estimated on rolling windows. Can you match the performance
of the Kalman filter with the simpler methodology in (a) by appropriately
choosing the length of the rolling window?
\end{problem}


\section{Topic: Another Interpretation for Independent Component Analysis (ICA)}
\begin{problem}
In class we considered the principally-independent component analysis method
which essentially was the truncated rank- $K$ SVD of a matrix X followed by an
ICA rotation of the left singular components:
$$
  \mathbf{X} \simeq \mathbf{U S V}^\top=\mathbf{U}_I \mathbf{S}_I \mathbf{V}_I
$$
where $\mathbf{U}_I=\mathbf{U} \mathbf{A}_I$ with
$\mathbf{A}_I=\underset{\mathbf{A}, \mathbf{A}^\top
    \mathbf{A}=I_K}{\operatorname{argmax}}\left|k_{\ell}(\mathbf{U A})\right|,
  k_{\ell}(\mathbf{G})$ being any centered cumulant of order $\ell \geq 3$ which
for all practical purposes can be considered a non-linear (activation)

function applied to each of the entries of $\mathbf{G}$. Furthermore the matrix
$\mathbf{V}_I$ was defined as $\mathbf{V}_I^\top:=\mathbf{D}^{-1}
  \mathbf{S}^{-1} \mathbf{A}_I \mathbf{S} \mathbf{V}^\top$ where $\mathbf{D}$
was chosen so that $\mathbf{V}_I$ has unimodular columns.
\begin{enumerate}[label=(\alph*)]
  \item Show that $\mathbf{D}$ is a diagonal matrix
  \item Show that $\mathbf{S}_I=\mathbf{S D}$ is diagonal such that
        $\operatorname{Tr}\left(\mathbf{S}_I^2\right)=\operatorname{Tr}\left(\mathbf{S}^2\right)$.
  \item Show that the method can be derived as the limit, $\lambda^2 \rightarrow
          0$, of the optimization
        $$
          \mathbf{U}_I, \mathbf{S}_I, \mathbf{V}_I=\underset{\mathbf{P}, \mathbf{Q}: \mathbf{P}^\top \mathbf{P}=\operatorname{diag}\left(\mathbf{Q}^\top \mathbf{Q}\right)=\mathbf{I}_k}{\operatorname{argmin}}\left\|\mathbf{X}-\mathbf{P R Q}^\top\right\|_F-\lambda^2\left|k_{\ell}(\mathbf{P})\right| .
        $$
  \item Show that an alternative objective function achieving the same result is
        $$
          \mathbf{U}_I, \mathbf{S}_I, \mathbf{V}_I=\underset{\mathbf{P}, \mathbf{Q}: \mathbf{P}^\top \mathbf{P}=\operatorname{diag}\left(\mathbf{Q}^\top \mathbf{Q}\right)=\mathbf{I}_k}{\operatorname{argmin}}\left\|\mathbf{X}-\mathbf{P R Q}^\top\right\|_F-\lambda^2 J(\mathbf{P}),
        $$
        where $J[\mathbf{x}]:=H\left[\mathbf{x}_{\text {gauss
          }}\right]-H[\mathbf{x}]$ is the negentropy and $J(\mathbf{P})$ is the
        sum of the negentropies of all the columns of $\mathbf{P}$.
\end{enumerate}
\end{problem}


























\end{document}

